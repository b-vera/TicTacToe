\section{Introducción}

\subsection{Contexto}
\noindent La necesidad de buscar un medio de entreteción ha sido parte de la sociedad desde tiempos remotos. Asi mismo es donde bajo este contexto en persia, alrededor del siglo V, se dio la creación del juego TicTacToe, Tres en Linea, Ceros y Cruces, entre otros nombres. Este juego fue difundido por los musulmanes y posteriormente llevado a europa por comerciantes italianos. El juego tuvo un rapido reconocimiento, aunque tuvo un periodo oscuro donde por una asociación de este con rituales paganos macabros, este estuvo prohibido para todos los cristianos de fe catolica.

\noindent El juego consiste en una cuadricula de 3x3 dentro de los cuales cada jugador debe colocar su simbolo una vez por turno y no debe ser sobre una casilla ya jugada. El ganador del juego es el jugador el cual logre realizar una linea recta o diagonal entre 3 de sus simbolos

\subsection{Problema}

\noindent En este laboratorio se le solicitó a los estudiantes de ingeniería informática de la Universidad de Santiago de Chile realizar mediante una serie de algoritmos en lenguaje C, un programa que reciba un archivo de entrada con instrucciones en formato MIPS y que permita generar una partida del juego gato, es decir, poder entregar quien fue el ganador de la partida, especificar el caso en que exista un empate o que las instrucciones hayan generado una jugada invalida ,todo esto con el fin de poder medir las habilidades adquiridas durante el desarrollo de éste curso y a la vez entregar la cantidad de veces que se ha utilizado cada etapa del camino de datos.

\subsection{Motivación}

La motivación de este laboratorio es poder aprender de manera didáctica, como proceden algunas de las distintas instrucciones propias del lenguaje MIPS frente a las etapas pertencientes al camino de datos de un procesador monociclo.

\subsection{Objetivos}
\subsubsection{Objetivo general}
\begin{enumerate}
    \item Simular el camino de datos de un procesador monociclo y determinar qué unidades funcionales del procesador son utilizadas en secuencias de instrucciones predefinida.
\end{enumerate}
\subsubsection{Objetivos específicos}
\begin{enumerate}
	\item Objetivo 3
	\item Objetivo 2
	\item Objetivo 1
\end{enumerate}

\subsection{Propuesta de solución}


\subsection{Herramientas}

\begin{enumerate}
    \item Herramienta 1
    \item Herramietna 2
\end{enumerate}

\subsection{Estructura del informe}
