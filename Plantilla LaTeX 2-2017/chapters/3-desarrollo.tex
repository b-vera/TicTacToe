\section{Desarollo}

\noindent Para el desarrollo de este laboratorio lo primero que se realiza es una estructura la cúal almacene de manera eficiente para un posterior trabajo la instrución a leer. Para esto se crea la estructura \textit{Move} la cual posee 4 punteros de tipo char, los cuales almacenarán especificamente la instrucción, 2 registros, y un valor numérico (el cual solo se empleará para los registros \textit{addi} y \textit{subi}).
Posterior a esto se realiza la lectura de cada linea del archivo de entrada, donde se revisa la primera palabra de la línea; si esta es una instrucción, dependiendo de su tipo, se realiza una lectura de los registro y datos asociados a este, para asi asociarlo a una estructura, la cual será almacenada posterior a este proceso en un arreglo que contendra todas las estructuras.

\noindent Una vez que todas las instrucciones se encuentren almacenadas de manera correcta en el arreglo de estructuras, se prosigue con la evaluación de cada una de estas y su respectiva función y a la declaración de un tablero, el cual consiste en un arreglo de caracteres con un largo de 9, el cual representara el tablero matricial. En primera instancia se revisan las primeras dos instrucciones y las define como los jugadores (jugadorX y jugadorO), ambas entragadas por una instrucción de tipo \textit{addi}. En la siguiente linea se verifica que se pida el almacenamiento para las jugadas a insertar. Posterior a estas 3 revisiones principales, se recorre el arreglo leyendo el tipo de instrucción que posee cada una de las instrucciones y según ésta, define que acción realizará. Antes de realizar la acción, se le realiza la consulta al tablero si la jugada, ya sea insertar un simbolo o eliminar un simbolo, exista y que a la vez, esta sea propia del jugador que quiere realizar la acción.En el caso que se realize una jugada no valida, tal como la inserción en alguna casilla no existente o la sobreescritura respecto una jugada del otro jugador, el programa entrega un mensaje que la partida esta incompleta y procede al cierre del . Al final de la lectura de todas las instrucciones del arreglo, se realiza una comprobación respecto a si hay un ganador (indicando el nombre de este en caso de existir) o si se genera un empate. Para realizar el cálculo de cuántas veces pasa cada instrucción por cada etapa en el camino de datos, se realizan sumas de valor 1 por cada etapa que realiza de manera constante cada instrucción, es decir, se asume que si es una instrucción de tipo \textit{addi} o \textit{subi}, estas pasarán por las 5 etapas del camino de datos, en cambio, una instrucción de tipo \textit{sw} solamente utiliza 4 etapas del camino de datos (se excluye Mem). A modo de finalización, el programa entrega dos documentos, uno llamado \textit{resultado.txt}, el cual posee la respuesta del ganador de la partida en caso de existir y el tablero con las jugadas realizadas, y un documento llamado \textit{Etapas.txt}, el cual entrega la cantidad de veces que cada instrucción pasa por cada etapa del camino de datos del procesador monociclo.


\begin{figure}[!ht]
	\centering
	\includegraphics[scale=0.4]{images/Ejemplo.png}
	\caption{Ejecución de las instrucciones en las distintas etapas del datapath.}
	\label{fig:ej}
\end{figure}
